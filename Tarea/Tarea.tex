\documentclass[11pt]{utalcaDoc}

\usepackage[activeacute,spanish]{babel}
\usepackage[utf8]{inputenc}
\usepackage{verbatim}
\usepackage{graphicx}
\usepackage{latexsym}
\usepackage{amsmath}
\usepackage{amssymb}
\usepackage{amsthm}
%\usepackage{anysize}
%\marginsize{2cm}{2cm}{1.7cm}{1.5cm}
\usepackage[top=3cm, bottom=2cm, left=1.8cm, right=1.8cm]{geometry}
\usepackage{url}
\usepackage{float}
\usepackage{amsfonts}

\usepackage{algpseudocode}



\usepackage{fancyhdr}

% aqui definimos el encabezado de las paginas pares e impares.
\lhead[Modelos Discretos -- T3]{Modelos Discretos -- T3}
%\chead[y1]{y2}
\rhead[Universidad de Talca]{Universidad de Talca}
\renewcommand{\headrulewidth}{0.5pt}



% aqui definimos el pie de pagina de las paginas pares e impares.
%\lfoot[d1]{e1}
%\cfoot[c1]{d2}
\rfoot[Victor Reyes, Pedro González]{Victor Reyes, Pedro González}
\renewcommand{\footrulewidth}{0.5pt}


\pagestyle{fancy} 


\title{{\bf Modelos Discretos}\\Tarea 1}
\author{Victor Reyes Medina,  Pedro González Meléndez}
\date{\today}

\begin{document}
\renewcommand{\figurename}{Figura~}
\renewcommand{\tablename}{Tabla~}
\renewcommand{\theenumii}{\arabic{enumii}}
\renewcommand{\labelenumii}{%
 \theenumi.\theenumii.
}
\maketitle

\section{Lógica proposicional}

\subsection{ } %1.1

$\\p:  \vartriangle$ ABC Isósceles \\
$q:  \vartriangle$ ABC Equilátero \\
$r:  \vartriangle$ ABC Equiangular\\

\begin{enumerate}
\item $p \to q$ Si el triángulo ABC es equilátero, entonces es isósceles.
\item $\neg p \to \neg q$ Si el triángulo ABC no es isósceles, entonces no es equiangular. 
\item $q \iff r$ El triángulo ABC es equilatero si y sólo si el triángulo es equiangular. 
\item $p \wedge \neg q$ Un triángulo ABC es isósceles y no equilátero.
\item $r \to p$ SI el triángulo ABC es quilátero, entonces es isósceles.
\end{enumerate}

\newpage 
\subsection{ } %1.2
\begin{enumerate}


\item
\[ \neg(p \lor \neg q) \to \neg p \]
Por ley de Morgan:
\[ (\neg p \wedge q) \to \neg p\]
\begin{tabular}{|c|c||c|c|c|}
\hline 
$p$ & \textbf{$q$} & \textbf{$(\neg p \wedge q)$} & \textbf{$\neg p$} &\textbf{$ (\neg p \wedge q) \to \neg p $} \\ 
\hline 
0 & 0 & 0 & 1 & 1 \\ 
\hline 
0 & 1 & 1 & 1 & 1 \\ 
\hline 
1 & 0 & 0 & 0 & 1 \\ 
\hline 
1 & 1 & 0 & 0 & 1 \\ 
\hline
\end{tabular} \ \ 
La expresión es una tautología.

\item  
\[ (p \to q) \to r\]
\begin{tabular}{|c|c|c||c|c|}
\hline 
$p$ & $q$ & $r$ & $(p \to q)$ &  $r$\\ 
\hline 
1 & 1 & 1 & 1 & 1 \\ 
\hline 
1 & 1 & 0 & 1 & 0 \\ 
\hline 
1 & 0 & 1 & 0 & 1 \\ 
\hline 
1 & 0 & 0 & 0 & 1 \\ 
\hline 
0 & 1 & 1 & 1 & 1 \\ 
\hline 
0 & 1 & 0 & 1 & 0 \\ 
\hline 
0 & 0 & 1 & 1 & 1 \\ 
\hline 
0 & 0 & 0 & 1 & 0 \\ 
\hline 
\end{tabular} \ \
La expresión es satisfacible.


\item
\[ (p \to q) \to (q \to p)\]

\begin{tabular}{|c|c||c|c|c|}
\hline
$p$ & $q$ & $(p \to q)$ & $(q \to p)$ & $(p \to q) \to (q \to p)$ \\
\hline 
0 & 0 & 1 & 1 & 1 \\ 
\hline 
0 & 1 & 1 & 0 & 0 \\ 
\hline 
1 & 0 & 0 & 1 & 1 \\ 
\hline 
1 & 1 & 1 & 1 & 1 \\  
\hline 
\end{tabular} \ \
La expresión es satisfacible.

\item
\[ ((p \to q) \wedge (q \to r)) \to (p \to r)\]
\begin{tabular}{|c|c|c||c|c|c|c|c|}
\hline
$p$ & $q$ & $r$ & $(p \to q)$ & $(q \to r)$ & $((p \to q) \wedge (q \to r))$ & $(p \to r)$ & $((p \to q) \wedge (q \to r)) \to (p \to r)$ \\
\hline 
1 & 1 & 1 & 1 & 1 & 1 & 1 & 1 \\ 
\hline 
1 & 1 & 0 & 1 & 0 & 0 & 0 & 1 \\ 
\hline 
1 & 0 & 1 & 0 & 1 & 0 & 1 & 1 \\ 
\hline 
0 & 1 & 1 & 1 & 1 & 1 & 1 & 1 \\ 
\hline 
1 & 0 & 0 & 0 & 1 & 0 & 0 & 1 \\ 
\hline 
0 & 1 & 0 & 1 & 0 & 0 & 1 & 1 \\ 
\hline 
0 & 0 & 1 & 1 & 1 & 1 & 1 & 1 \\ 
\hline 
0 & 0 & 0 & 1 & 1 & 1 & 1 & 1 \\ 
\hline 
\end{tabular} \\ 
La expresión es una tautología.



\end{enumerate}
\newpage
\subsection{ } %1.3

\newpage
\section{Lógica de primer orden}


\subsection{ } %2.1

\newpage
\subsection{ } %2.2
 
\newpage
\subsection{ } %2.3


\end{document}
