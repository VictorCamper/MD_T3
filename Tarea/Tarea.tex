\documentclass[11pt]{utalcaDoc}

\usepackage[activeacute,spanish]{babel}
\usepackage[utf8]{inputenc}
\usepackage{verbatim}
\usepackage{graphicx}
\usepackage{latexsym}
\usepackage{amsmath}
\usepackage{amssymb}
\usepackage{amsthm}
%\usepackage{anysize}
%\marginsize{2cm}{2cm}{1.7cm}{1.5cm}
\usepackage[top=2.7cm, bottom=2cm, left=1.8cm, right=1.8cm]{geometry}
\usepackage{url}
\usepackage{float}
\usepackage{amsfonts}

\usepackage{algpseudocode}



\usepackage{fancyhdr}

% aqui definimos el encabezado de las paginas pares e impares.
\lhead[Modelos Discretos -- T3]{Modelos Discretos -- T3}
%\chead[y1]{y2}
\rhead[Universidad de Talca]{Universidad de Talca}
\renewcommand{\headrulewidth}{0.5pt}



% aqui definimos el pie de pagina de las paginas pares e impares.
%\lfoot[d1]{e1}
%\cfoot[c1]{d2}
\rfoot[Victor Reyes, Pedro González]{Victor Reyes, Pedro González}
\renewcommand{\footrulewidth}{0.5pt}


\pagestyle{fancy} 


\title{{\bf Modelos Discretos}\\Tarea 1}
\author{Victor Reyes Medina,  Pedro González Meléndez}
\date{\today}

\begin{document}
\renewcommand{\figurename}{Figura~}
\renewcommand{\tablename}{Tabla~}
\renewcommand{\theenumii}{\arabic{enumii}}
\renewcommand{\labelenumii}{%
 %\theenumi.\theenumii.
  \theenumii.
}
\maketitle

\section{Lógica proposicional}

\subsection{ } %1.1

$\\p:  \vartriangle$ ABC Isósceles \\
$q:  \vartriangle$ ABC Equilátero \\
$r:  \vartriangle$ ABC Equiangular\\

\begin{enumerate}
\item $p \to q$ Si el triángulo ABC es equilátero, entonces es isósceles.
\item $\neg p \to \neg q$ Si el triángulo ABC no es isósceles, entonces no es equiangular. 
\item $q \iff r$ El triángulo ABC es equilatero si y sólo si el triángulo es equiangular. 
\item $p \wedge \neg q$ Un triángulo ABC es isósceles y no equilátero.
\item $r \to p$ SI el triángulo ABC es quilátero, entonces es isósceles.
\end{enumerate}

\newpage 
\subsection{ } %1.2
\begin{enumerate}


\item
\[ \neg(p \lor \neg q) \to \neg p \]
Por ley de Morgan:
\[ (\neg p \wedge q) \to \neg p\]
\begin{tabular}{|c|c||c|c|c|}
\hline 
$p$ & \textbf{$q$} & \textbf{$(\neg p \wedge q)$} & \textbf{$\neg p$} &\textbf{$ (\neg p \wedge q) \to \neg p $} \\ 
\hline 
0 & 0 & 0 & 1 & 1 \\ 
\hline 
0 & 1 & 1 & 1 & 1 \\ 
\hline 
1 & 0 & 0 & 0 & 1 \\ 
\hline 
1 & 1 & 0 & 0 & 1 \\ 
\hline
\end{tabular} \ \ 
La expresión es una tautología.

\item  
\[ (p \to q) \to r\]
\begin{tabular}{|c|c|c||c|c|}
\hline 
$p$ & $q$ & $r$ & $(p \to q)$ &  $r$\\ 
\hline 
1 & 1 & 1 & 1 & 1 \\ 
\hline 
1 & 1 & 0 & 1 & 0 \\ 
\hline 
1 & 0 & 1 & 0 & 1 \\ 
\hline 
1 & 0 & 0 & 0 & 1 \\ 
\hline 
0 & 1 & 1 & 1 & 1 \\ 
\hline 
0 & 1 & 0 & 1 & 0 \\ 
\hline 
0 & 0 & 1 & 1 & 1 \\ 
\hline 
0 & 0 & 0 & 1 & 0 \\ 
\hline 
\end{tabular} \ \
La expresión es satisfacible.


\item
\[ (p \to q) \to (q \to p)\]

\begin{tabular}{|c|c||c|c|c|}
\hline
$p$ & $q$ & $(p \to q)$ & $(q \to p)$ & $(p \to q) \to (q \to p)$ \\
\hline 
0 & 0 & 1 & 1 & 1 \\ 
\hline 
0 & 1 & 1 & 0 & 0 \\ 
\hline 
1 & 0 & 0 & 1 & 1 \\ 
\hline 
1 & 1 & 1 & 1 & 1 \\  
\hline 
\end{tabular} \ \
La expresión es satisfacible.

\item
\[ ((p \to q) \wedge (q \to r)) \to (p \to r)\]
\begin{tabular}{|c|c|c||c|c|c|c|c|}
\hline
$p$ & $q$ & $r$ & $(p \to q)$ & $(q \to r)$ & $((p \to q) \wedge (q \to r))$ & $(p \to r)$ & $((p \to q) \wedge (q \to r)) \to (p \to r)$ \\
\hline 
1 & 1 & 1 & 1 & 1 & 1 & 1 & 1 \\ 
\hline 
1 & 1 & 0 & 1 & 0 & 0 & 0 & 1 \\ 
\hline 
1 & 0 & 1 & 0 & 1 & 0 & 1 & 1 \\ 
\hline 
0 & 1 & 1 & 1 & 1 & 1 & 1 & 1 \\ 
\hline 
1 & 0 & 0 & 0 & 1 & 0 & 0 & 1 \\ 
\hline 
0 & 1 & 0 & 1 & 0 & 0 & 1 & 1 \\ 
\hline 
0 & 0 & 1 & 1 & 1 & 1 & 1 & 1 \\ 
\hline 
0 & 0 & 0 & 1 & 1 & 1 & 1 & 1 \\ 
\hline 
\end{tabular} \\ 
La expresión es una tautología.



\end{enumerate}
\newpage
\subsection{ } %1.3
\begin{enumerate}
\item \[ \]
a)
	\[ p \to (q \wedge r) \equiv (p \to q) \wedge (p \to r)\]
%Por el lado izquierdo:\\
\begin{tabular}{|c|c|c||c|c|c|}
\hline 
$p$ & $q$ & $r$ & $p$ & $(q \wedge r)$ & $p \to (q \wedge r)$ \\ 
\hline 
0 & 0 & 0 & 0 & 0 & 1 \\ 
\hline 
0 & 0 & 1 & 0 & 0 & 1 \\ 
\hline 
0 & 1 & 0 & 0 & 0 & 1 \\ 
\hline 
0 & 1 & 1 & 0 & 1 & 1 \\ 
\hline 
1 & 0 & 0 & 1 & 0 & 0 \\ 
\hline 
1 & 0 & 1 & 1 & 0 & 0 \\ 
\hline 
1 & 1 & 0 & 1 & 0 & 0 \\ 
\hline 
1 & 1 & 1 & 1 & 1 & 1 \\ 
\hline 
\end{tabular} 
%\\ \\ \ \
%Por el lado derecho: \\
\begin{tabular}{|c|c|c||c|c|c|}
\hline 
$p$ & $q$ & $r$ & $(p \to q)$ & $(p \to r)$ & $(p \to q) \wedge (p \to r)$ \\ 
\hline 
0 & 0 & 0 & 1 & 1 & 1 \\ 
\hline 
0 & 0 & 1 & 1 & 1 & 1 \\ 
\hline 
0 & 1 & 0 & 1 & 1 & 1 \\ 
\hline 
0 & 1 & 1 & 1 & 1 & 1 \\ 
\hline 
1 & 0 & 0 & 0 & 0 & 0 \\ 
\hline 
1 & 0 & 1 & 0 & 1 & 0 \\ 
\hline 
1 & 1 & 0 & 1 & 0 & 0 \\ 
\hline 
1 & 1 & 1 & 1 & 1 & 1 \\ 
\hline 
\end{tabular} 
\\ \\ \ \ 
Las fórmulas son lógicamente equivalentes.
\[ \]
b)
	\[ (p \lor q) \to r \equiv (p \to r) \wedge (q \to r) \]
\begin{tabular}{|c|c|c||c|c|c|}
\hline 
$p$ & $q$ & $r$ & $(p \lor q)$ & $r$ & $(p \lor q) \to r$ \\ 
\hline 
0 & 0 & 0 & 0 & 0 & 1 \\ 
\hline 
0 & 0 & 1 & 0 & 1 & 1 \\ 
\hline 
0 & 1 & 0 & 1 & 0 & 0 \\ 
\hline 
0 & 1 & 1 & 1 & 1 & 1 \\ 
\hline 
1 & 0 & 0 & 1 & 0 & 0 \\ 
\hline
1 & 0 & 1 & 1 & 1 & 1 \\ 
\hline 
1 & 1 & 0 & 1 & 0 & 0 \\ 
\hline 
1 & 1 & 1 & 1 & 1 & 1 \\ 
\hline 
\end{tabular} 
%\\
\begin{tabular}{|c|c|c||c|c|c|}
\hline 
$p$ & $q$ & $r$ & $(p \to r)$ & $(q \to r)$ & $(p \to r) \wedge (q \to r)$ \\ 
\hline 
0 & 0 & 0 & 1 & 1 & 1 \\ 
\hline 
0 & 0 & 1 & 1 & 1 & 1 \\ 
\hline 
0 & 1 & 0 & 1 & 0 & 0 \\ 
\hline 
0 & 1 & 1 & 1 & 1 & 1 \\ 
\hline
1 & 0 & 0 & 0 & 1 & 0 \\ 
\hline 
1 & 0 & 1 & 1 & 1 & 1 \\ 
\hline 
1 & 1 & 0 & 0 & 0 & 0 \\ 
\hline 
1 & 1 & 1 & 1 & 1 & 1 \\ 
\hline 
\end{tabular} 
\\ \\ \ \
Las fórmulas son lógicamente equivalentes.
\[ \]
c)
\[ p \to (q \lor r) \equiv \neg r \to (p \to q) \]
\begin{tabular}{|c|c|c||c|c|c|}
\hline 
$p$ & $q$ & $r$ & $p$ & $(q \lor r)$ & $p \to (q \lor r)$ \\ 
\hline 
0 & 0 & 0 & 0 & 0 & 1 \\ 
\hline 
0 & 0 & 1 & 0 & 1 & 1 \\ 
\hline 
0 & 1 & 0 & 0 & 1 & 1 \\ 
\hline 
0 & 1 & 1 & 0 & 1 & 1 \\ 
\hline 
1 & 0 & 0 & 1 & 0 & 0 \\ 
\hline 
1 & 0 & 1 & 1 & 1 & 1 \\ 
\hline 
1 & 1 & 0 & 1 & 1 & 1 \\ 
\hline
1 & 1 & 1 & 1 & 1 & 1 \\ 
\hline 
\end{tabular} 
% \\
\begin{tabular}{|c|c|c||c|c|c|}
\hline 
$p$ & $q$ & $r$ & $\neg r$ & $(p \to q)$ & $\neg r \to (p \to q)$ \\
\hline 
0 & 0 & 0 & 1 & 1 & 1 \\ 
\hline 
0 & 0 & 1 & 0 & 1 & 1 \\ 
\hline 
0 & 1 & 0 & 1 & 1 & 1 \\ 
\hline 
0 & 1 & 1 & 0 & 1 & 1 \\ 
\hline 
1 & 0 & 0 & 1 & 0 & 0 \\ 
\hline 
1 & 0 & 1 & 0 & 0 & 1 \\ 
\hline 
1 & 1 & 0 & 1 & 1 & 1 \\ 
\hline 
1 & 1 & 1 & 0 & 1 & 1 \\ 
\hline 
\end{tabular}
\\ \\ \ \
Las fórmulas son lógicamente equivalentes.

\newpage
\item

\[ p \to (q \lor r) \equiv (p \wedge \neg q) \to r \]
Aplicamos sustitución de implicación en ambos lados $(x \to y) = (\neg x \lor y)$
\[ \neg p \lor (q \lor r) \equiv \neg (p \wedge \neg q) \lor r \]
Aplicamos asocitividad en el lado izquierdo $(x \lor (y \lor z)) = ((x \lor y) \lor z)$ \\
Y Ley de Morgan en el lado derecho $\neg(x \wedge y) = (\neg x \lor \neg y)$
\[ (\neg p \lor q) \lor r \equiv (\neg p \lor q) \lor r \]
Se tiene en ambos lados la misma fórmula lógica, por lo que su equivalencia lógica queda demostrada
\end{enumerate}
\newpage
\section{Lógica de primer orden}


\subsection{ } %2.1

\begin{enumerate}
\item 

\begin{align*}
&Hora(t) : \text{Horario de atención (entre 09:00 y 18:00)}\\
&AbiertaB(b) : \text{Bodega b abierta}\\
&Guardar(p,b) : \text{Persona p guarda materiales en Bodega b}\\
&Sacar(p,b) : \text{Persona p saca materiales de Bodega b}\\
&Llave(p,b) : \text{Persona p tiene llaves de Bodega b}\\
&Capataz(p) : \text{Persona p es capataz}\\
&\forall b \in Bodegas \text{ y } t_a = \text{hora actual} \quad (Hora(t_a) \to AbiertaB(b))\\
&\forall p \in Personas, b \in Bodegas \text{ y } t_a = \text{hora actual} \quad (Guardar(p,b,t) \iff Hora(t_a) \lor Llave(p,b))\\
&\forall p \in Personas, b \in Bodegas \text{ y } t_a = \text{hora actual} \quad (Sacar(p,b,t) \iff Hora(t_a) \lor Llave(p,b))\\
&\exists p \in Personas, \forall b \in Bodegas \quad (Llave(p,b) \iff Capataz(p))
\end{align*}


\item Se tiene:



$Guardar(Pedro,Bodega1,20:00)$

\rule[0.1cm]{10cm}{0.1pt}

$Capataz(Pedro)$

a) Demostración directa.
\begin{enumerate}
\item Para $p = $ Pedro, $b = $ Bodega1 y $t_a = $ 20:00.
\item $Guardar(Pedro,Bodega1,20:00)$ (Axioma).
\item $\underbrace{Hora(20:00)}_{(1)} \lor \underbrace{Llave(Pedro, Bodega1)}_{(2)}$.
\item $(1)$ es falso ya que $20:00$ no es parte de los horarios de atención.
\item De $(2)$ se tiene que $Capataz(Pedro)$ (Axioma).
\item Finalmente, Pedro es Capataz.
\end{enumerate}

b) Demostración por contradicción.
\begin{enumerate}
\item Para $p = $ Pedro, $b = $ Bodega1 y $t_a = $ 20:00.
\item Asumiendo $\neg(Capataz(Pedro))$ y $Guardar(Pedro,Bodega1,20:00)$. Pedro no es Capataz, pero guardó materiales fuera del horario de atención.
\item Se tiene que (por equivalencia) $\neg (Llave(Pedro, Bodega1))$, Pedro no tiene las llaves de la Bodega1.
\item Reemplazando en el axioma instanciado \\ $Guardar(Pedro,Bodega1,20:00) \iff Hora(20:00) \lor \neg (Llave(Pedro, Bodega1)) $.
\item Se sabe que $Hora(20:00) = 0$, por lo que el lado derecho de la equivalencia es falso.
\item Por implicancia, se tiene que $\neg(Guardar(Pedro,Bodega1,20:00))$, lo que entra en contradicción con lo expuesto en el paso 2.
\item Finalmente, por contradicción se demuestra que Pedro es Capataz.
\end{enumerate}


\end{enumerate}

\newpage
\subsection{ } %2.2
 
\newpage
\subsection{ } %2.3


\end{document}
